\newgeometry{margin=5cm}
\thispagestyle{empty}
\begin{multicols}{2}
[
\section*{Cuvânt înainte,}
]
\footnotesize {
Documentul de față își propune să vă ajute pe parcursul desfășurării proiectului \textit{"Line Follower"} organizat de \textit{Freescale România} în parteneriat cu \textit{Junior Achievement România}.

Dorim să vă informăm, de la bun început, de faptul că limbajul sau anumiți termeni ce apar în cadrul acestui document s-ar putea să nu vă fie familiari. Chiar dacă acest lucru se întâmplă, nu trebuie să vă panicați pentru că multe din aceste informații vă vor fi prezentate la sedințele tehnice pe care le veți avea cu mentorii. De fapt, prima ședință chiar asta îți propune: să aducă întreagă echipa la același nivel de cunoștințe și să aibă un vocabular adecvat. În celelalte trei ședințe ne vom ocupa de scris cod și optimizat soluția, dar începutul e cel mai important.

Vă recomandăm să citiți acest document înainte de a avea loc discuțiile cu mentorii. În acest fel, veți putea să vă faceți o idee despre ceea ce vom realiza, iar dacă ceva nu vă este clar, vă invităm să pregătiți o listă de întrebări pentru mentorii voștri.

Ce vei găsi în materialul de față? În primul rând, se vor regăsi explicații teoretice / fizice prin care se demonstrează modul de funcționare a unor module hardware și cum acestea pot fi programate folosind o unealtă ajutătoare. În al doilea rând, vei fi introdus în lumea sistemelor embedded, vei învăța cum poți controla un dispozitiv prin intermediul unui algoritm transpus într-un limbaj de programare, precum C.

}
\end{multicols}
\begin{flushright}\textit{Mult succes,\\Mentorii Freescale}\end{flushright}
\restoregeometry