\chapter{Sisteme încorporate}

Termenul de \textit{sistem încorporat} reprezintă traducerea în limba română a \textit{embedded system} și semnifică un calculator de mici dimensiuni bazat pe un microprocesor, util în vederea îndeplinirii unei sarcini anume (foarte rar avem de-a face cu mai multe sarcini). În plus, se pune accentul pe posibilitatea procesării datelor și luarea unor decizii în timp real. Dacă la un calculatator personal este acceptabilă o întârziere de 2-3 secunde atunci când are loc încărcarea unei aplicații, în lumea embedded un lag de acest gen, poate conduce la pierderea ireversibilă a stării curente.

Un astfel de sistem poate acoperi o gamă largă de aplicații, fiind constituit din mai multe componente mecanice și electronice ce sunt interconectate și administrate de un modul cheie - un microcontroller. Acesta din urmă nu este altceva decât creierul, elementul prin care este insuflată inteligență dispozitivului pe care îl creăm. Ca și definiție, microcontroller-ul este un chip (circuit integrat) ce încorporează o unitate centrală (CPU) și o memorie împreună cu resurse care-i permit interacțiunea cu mediul exterior (interacțiunea cu mediul exterior se va realiza prin intermediul unor senzori).

Microcontrollerele sunt utilizate în foarte multe domenii, dintre care am enumera: industria de automobile (controlul aprinderii motorului, climatizare, diagnoză, sisteme de alarmă, computer de bord), în așa zisa electronică de consum (sisteme audio, televizoare, camere video și videocasetofoane, telefonie mobilă, GPS-uri, jocuri electronice), în aparatura electrocasnică (mașini de spălat, frigidere, cuptoare cu microunde, aspiratoare), în controlul mediului și climatizare (sere, locuințe, hale industriale), în industria aerospațială, în mijloacele moderne de măsurare - instrumentație (aparate de măsură, senzori și traductoare inteligente) cât și în medicină. Ca un exemplu din industria de automobile, numai la nivelul anului 1999, un BMW seria 7 utiliza 65 de microcontrolere, iar un Mercedes din clasa S utiliza 63 de microcontrolere. Practic, deși am prezentat ca exemple concrete numai sisteme robotice și mecatronice, este foarte greu de găsit un domeniu de aplicații în care să nu se utilizeze microcontrolerele.

\section{Super Loop Architecture}
Majoritatea sistemelor de tip embedded sunt caracterizate de constrângeri de timp - anumite activități trebuie efectuate la momente critice / bine stabilite de timp, ordinea în care sunt efectuate activitățile putând conta, de asemenea. O arhitectură ce răspunde bine acestor cerinte este \textbf{"Super Loop Architecture"}.
Această arhitectură constă într-o buclă infinită ce include toate sarcinile pe care sistemul trebuie să le execute:

\lstinputlisting[caption=Exemplu de arhitectură Super Loop, style=customc]{sources/superLoop.lst}
Se observă că inițializarea sistemului se face în afara buclei principale, iar în buclă se efectuază, în această ordine: citirea intrărilor, calculul unor valori și setarea ieșirilor. De cele mai multe ori, viteza procesorului embedded este mai mare decât timpul necesar efectuării operațiilor din sbucla principală. Sa luăm ca exemplu un sistem ce permite efectuarea unei iterații într-o milisecundă, iar verificarea intrărilor și scrierea ieșirilor pot fi făcute o singură dată pe secundă. În acest caz, bucla va fi parcursa de 1000 ori pe secunda, dar doar o dată va citi intrările / scrie ieșiri. Putem așadar modifica superbucla pentru a ne folosi de o întârziere:

\lstinputlisting[caption=Exemplu de arhitectură Super Loop cu întâziere pe fiecare iterație, style=customc]{sources/superLoopDelayed.lst}
Dacă sistemul permite modul \textit{power-save}, atunci operația de întârziere poate beneficia de acest mod pentru a salva din puterea consumată atunci când ne aflăm în așteptare.
